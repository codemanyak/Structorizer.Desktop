\documentclass[a4paper,10pt]{article}

\usepackage[inoutnumbered]{algorithm2e}
\usepackage{ngerman}
\usepackage{amsmath}

\DeclareMathOperator{\oprdiv}{div}
\DeclareMathOperator{\oprshl}{shl}
\DeclareMathOperator{\oprshr}{shr}
\SetKwInput{Input}{input}
\SetKwInput{Output}{output}
\SetKwBlock{Parallel}{parallel}{end}
\SetKwFor{Thread}{thread}{:}{end}
\SetKwBlock{Try}{try}{end}
\SetKwFor{Catch}{catch (}{)}{end}
\SetKwBlock{Finally}{finally}{end}
\SetKwProg{Prog}{Program}{ }{end}
\SetKwProg{Proc}{Procedure}{ }{end}
\SetKwProg{Func}{Function}{ }{end}
\SetKwProg{Incl}{Includable}{ }{end}

\SetKw{preLeave}{leave}
\SetKw{preExit}{exit}
\SetKw{preThrow}{throw}
\PrintSemicolon
\title{Structorizer LaTeX pseudocode Export of FileApiGroupTest.arrz}
\author{Kay G"urtzig}
\date{09.06.2021}

\begin{document}
\LinesNumbered

\begin{procedure}
\caption{drawBarChart(values, nValues)}
\tcc{ Draws a bar chart from the array "{}values"{} of size nValues. }
\tcc{ Turtleizer must be activated and will scale the chart into a square of }
\tcc{ 500 x 500 pixels }
\tcc{ Note: The function is not robust against empty array or totally equal values. }
\Proc{\FuncSty{drawBarChart(}\ArgSty{values, nValues}\FuncSty{)}}{
\KwData{\(values\): array of double}
\KwData{\(nValues\): ?}
  \tcc{ Used range of the Turtleizer screen }
  \(const\ xSize\leftarrow{}500\)\;
  \(const\ ySize\leftarrow{}500\)\;
  \(kMin\leftarrow{}0\)\;
  \(kMax\leftarrow{}0\)\;
  \For{\(k\leftarrow 1\) \KwTo \(nValues-1\) \textbf{by} \(1\)}{
    \eIf{\(values[k]>values[kMax]\)}{
      \(kMax\leftarrow{}k\)\;
    }{
      \If{\(values[k]<values[kMin]\)}{
        \(kMin\leftarrow{}k\)\;
      }
    }
  }
  \(valMin\leftarrow{}values[kMin]\)\;
  \(valMax\leftarrow{}values[kMax]\)\;
  \(yScale\leftarrow{}valMax*1.0/(ySize-1)\)\;
  \(yAxis\leftarrow{}ySize-1\)\;
  \If{\(valMin<0\)}{
    \eIf{\(valMax>0\)}{
      \(yAxis\leftarrow{}valMax*ySize*1.0/(valMax-valMin)\)\;
      \(yScale\leftarrow{}(valMax-valMin)*1.0/(ySize-1)\)\;
    }{
      \(yAxis\leftarrow{}1\)\;
      \(yScale\leftarrow{}valMin*1.0/(ySize-1)\)\;
    }
  }
  \tcc{ draw coordinate axes }
  \(gotoXY(1,ySize-1)\)\;
  \(forward(ySize-1)\)\;
  \(penUp()\)\;
  \(backward(yAxis)\)\;
  \(right(90)\)\;
  \(penDown()\)\;
  \(forward(xSize-1)\)\;
  \(penUp()\)\;
  \(backward(xSize-1)\)\;
  \(stripeWidth\leftarrow{}xSize/nValues\)\;
  \For{\(k\leftarrow 0\) \KwTo \(nValues-1\) \textbf{by} \(1\)}{
    \(stripeHeight\leftarrow{}values[k]*1.0/yScale\)\;
    \Switch{k\bmod\ 3}{
      \Case{0}{
        \(setPenColor(255,0,0)\)\;
      }
      \Case{1}{
        \(setPenColor(0,255,0)\)\;
      }
      \Case{2}{
        \(setPenColor(0,0,255)\)\;
      }
    }
    \(fd(1)\)\;
    \(left(90)\)\;
    \(penDown()\)\;
    \(fd(stripeHeight)\)\;
    \(right(90)\)\;
    \(fd(stripeWidth-1)\)\;
    \(right(90)\)\;
    \(forward(stripeHeight)\)\;
    \(left(90)\)\;
    \(penUp()\)\;
  }
}
\end{procedure}


\begin{function}
\caption{readNumbers(fileName, numbers, maxNumbers)}
\tcc{ Tries to read as many integer values as possible upto maxNumbers }
\tcc{ from file fileName into the given array numbers. }
\tcc{ Returns the number of the actually read numbers. May cause an exception. }
\Func{\FuncSty{readNumbers(}\ArgSty{fileName, numbers, maxNumbers}\FuncSty{)}:integer}{
\KwData{\(fileName\): string}
\KwData{\(numbers\): array of integer}
\KwData{\(maxNumbers\): integer}
\KwResult{integer}
  \(nNumbers\leftarrow{}0\)\;
  \(fileNo\leftarrow{}fileOpen(fileName)\)\;
  \If{\(fileNo\leq\ 0\)}{
    \preThrow{\(\)"{}File could not be opened!"{}\(\)}\;
  }
  \Try{
    \While{\(!fileEOF(fileNo)\wedge\ nNumbers<maxNumbers\)}{
      \(number\leftarrow{}fileReadInt(fileNo)\)\;
      \(numbers[nNumbers]\leftarrow{}number\)\;
      \(nNumbers\leftarrow{}nNumbers+1\)\;
    }
  }
  \Catch{error}{
    \preThrow{\(\)}\;
  }
  \Finally{
    \(fileClose(fileNo)\)\;
  }
  \Return{\(nNumbers\)};
}
\end{function}


\begin{algorithm}
\caption{DrawRandomHistogram}
\SetKwFunction{FnreadNumbers}{readNumbers}
\SetKwFunction{FndrawBarChart}{drawBarChart}
\tcc{ Reads a random number file and draws a histogram accotrding to the }
\tcc{ user specifications }
\Prog{\FuncSty{DrawRandomHistogram}}{
  \(fileNo\leftarrow{}-10\)\;
  \Repeat{\(fileNo>0\vee\ file\_name=\)"{}"{}\(\)}{
    \Input{\(\)"{}Name/path\ of\ the\ number\ file"{}\(file\_name\)}
    \(fileNo\leftarrow{}fileOpen(file\_name)\)\;
  }
  \If{\(fileNo>0\)}{
    \(fileClose(fileNo)\)\;
    \Input{\(\)"{}number\ of\ intervals"{}\(,nIntervals\)}
    \tcc{ Initialize the interval counters }
    \For{\(k\leftarrow 0\) \KwTo \(nIntervals-1\) \textbf{by} \(1\)}{
      \(count[k]\leftarrow{}0\)\;
    }
    \tcc{ Index of the most populated interval }
    \(kMaxCount\leftarrow{}0\)\;
    \(numberArray\leftarrow{}\{\}\)\;
    \(nObtained\leftarrow{}0\)\;
    \Try{
      \(nObtained\leftarrow{}\FnreadNumbers(file\_name,numberArray,10000)\)\;
    }
    \Catch{failure}{
      \Output{\(failure\)}
    }
    \eIf{\(nObtained>0\)}{
      \(min\leftarrow{}numberArray[0]\)\;
      \(max\leftarrow{}numberArray[0]\)\;
      \For{\(i\leftarrow 1\) \KwTo \(nObtained-1\) \textbf{by} \(1\)}{
        \eIf{\(numberArray[i]<min\)}{
          \(min\leftarrow{}numberArray[i]\)\;
        }{
          \If{\(numberArray[i]>max\)}{
            \(max\leftarrow{}numberArray[i]\)\;
          }
        }
      }
      \tcc{ Interval width }
      \(width\leftarrow{}(max-min)*1.0/nIntervals\)\;
      \For{\(i\leftarrow 0\) \KwTo \(nObtained-1\) \textbf{by} \(1\)}{
        \(value\leftarrow{}numberArray[i]\)\;
        \(k\leftarrow{}1\)\;
        \While{\(k<nIntervals\wedge\ value>min+k*width\)}{
          \(k\leftarrow{}k+1\)\;
        }
        \(count[k-1]\leftarrow{}count[k-1]+1\)\;
        \If{\(count[k-1]>count[kMaxCount]\)}{
          \(kMaxCount\leftarrow{}k-1\)\;
        }
      }
      \(\FndrawBarChart(count,nIntervals)\)\;
      \Output{\(\)"{}Interval\ with\ max\ count:\ "{}\(,kMaxCount,\)"{}\ ("{}\(,count[kMaxCount],\)"{})"{}\(\)}
      \For{\(k\leftarrow 0\) \KwTo \(nIntervals-1\) \textbf{by} \(1\)}{
        \Output{\(count[k],\)"{}\ numbers\ in\ interval\ "{}\(,k,\)"{}\ ("{}\(,min+k*width,\)"{}\ ...\ "{}\(,min+(k+1)*width,\)"{})"{}\(\)}
      }
    }{
      \Output{\(\)"{}No\ numbers\ read."{}\(\)}
    }
  }
}
\end{algorithm}

\end{document}
